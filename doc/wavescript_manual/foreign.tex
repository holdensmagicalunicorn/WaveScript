
\chapter{Foreign (C/C++) Interface}

The WaveScript compiler provides a facility for calling external
(foreign) functions written in C or C++.  The primary reasons for this
are two-fold.

\begin{enumerate}
\item We wish to reuse existing libraries without modification (e.g. Gnu
  Scientific Library, FFTW, etc).

\item We wish to add new stream sources and sinks --- for network communication, disk access and so
  on --- without modifying the WaveScript compiler itself.  Also, we
  frequently want to add support for new hardware data-sources
  (sensors).
  
%In the future we will have a {\tt foreign\_box} syntax similar to {\tt
%  foreign}, but allowing the user to write a WSSource class usable by
%  the WaveScope engine.}.
\end{enumerate}

There are three WaveScript primitives used to interface with foreign
code.  The {\tt foreign} primitive registers a single C function with
WaveScript.  Alternatively, {\tt foreign\_source} imports a stream of
values from foreign code.  It does this by providing a C function that
can be called to add a single tuple to the stream.  Thus we can call
from WaveScript into C and from C into WaveScript.  The third
primitive is {\tt inline\_C}.  It allows WaveScript to generate
arbitrary C code at compile time which is then linked into the final
stream query.  We can of course call into the C code we generate from
WaveScript (or it can call into us).
%Thus we can generate C code and then call into the code
%we generated (or have it call into us).




\section{Foreign functions}

The basic foreign function primitive is called as follows: ``{\tt
foreign({\em function-name}, {\em file-list})}''.  Like any other
primitive function, {\tt foreign} can be used anywhere within a
WaveScript program.  It returns a WaveScript function representing the
corresponding C function of the given name.  The only restriction is
that any call to the {\tt foreign} primitive {\em must} have a type
annotation.  The type annotation lets WaveScript type-check the
program, and tells the WaveScript compiler how to convert (if
necessary) WaveScript values into C-values when the foreign function
is called.  

The second argument is a list of {\em dependencies}---files that must
be compiled/linked into the query for the foreign function to be
available.
%
For example, the following would import a function ``foo'' from ``foo.c''.

\begin{wscode}
c\_foo :: Int -> Int = foreign("foo", ["foo.c"])
\end{wscode}

%Note that, as when reading data from a file, the compiler has no way
%to know the type of the external function, so an explicit type
%annotation is required.

Currently C-code can be loaded from source files ({\tt .c}, {\tt
  .cpp}) or object files ({\tt .o}, {\tt .a}, {\tt .so}).  When
  loading from object files, it's necessary to also include a header
  ({\tt .h}, {\tt .hpp}).  For example:

\begin{verbatim}
c_bar = 
   (foreign("bar", ["bar.h", "bar.a"]) 
    :: Int -> Int)
\end{verbatim}

Of course, you may want to import many functions from the same file or
library.  WaveScript uses a very simple rule.  If a file has already
been imported once, repeated imports are suppressed.  (This goes for
source and object files.)  Also, if you try to import multiple files with
the same basename (e.g. ``bar.o'' and ``bar.so'') the behavior is
currently undefined.





\section{Foreign Sources}

A call to register a foreign source has the same form as for a foreign
function: ``{\tt foreign\_source({\em function-name}, {\em
file-list})}''.  However, in this case the {\em function-name} is the
name of the function being {\em exported}.  The call to {\tt
foreign\_source} will return a stream of incoming values.  It must be
annotated with a type of the form {\tt Stream} $T$, where $T$ is a
type that supports marshaling from C code.
%
We call the function exported to C an {\em entrypoint}.  It will take
a single argument, convert it to the WaveScript representation, and
fire off a tuple as one element of the input stream.  The return
behavior of this entrypoint is determined by the scheduling policy
employed by the particular WaveScope backend.  For example, it may
follow the tuple through in a depth-first traversal of the stream
graph, returning only when there is no further processing.  Or the
entrypoint function may return immediately, merely enqueuing the tuple
for later processing.  The entrypoint will return an integer error
code, which is zero if the WaveScope process is still in a healthy
state at the time the entrypoint returns.  Note that a zero
return-code does not guarantee that an error will not be encountered
as the data is further processed, but before the entrypoint is called
again.

Currently, using multiple foreign sources is supported (i.e. multiple
entrypoints into WaveScript).  However, if using foreign sources, you
cannot also use built-in WaveScript ``timer'' sources.  When driving
the system from foreign sources, the entire WaveScript system becomes
just a set of functions that can be called from C.  The system is
dormant until one of these entrypoints is called.

Because the main thread of control belongs to the foreign C code,
there is another convention that must be followed.  The user must implement
{\em three} functions that WaveScript uses to initialize, start up the
system, and handle errors respectively.

\begin{verbatim}
  void wsinit(int argc, char** argv)
  void wsmain(int argc, char** argv)
  void wserror(const char*)
\end{verbatim}

{\tt Wsinit} is called at startup, before any WaveScript code runs
(e.g. before {\tt state\{\}} blocks are initialized, and even before
constants are allocated).  {\tt Wsmain} is called when the WaveScript
dataflow graph is finished initialing and is ready to receive data.
{\tt Wsmain} should control all subsequent acquisition of data, and
feed data into WaveScript through the registered {\tt foreign\_source}
functions.  {\tt Wserror} is called when WaveScope reaches an error,
it may choose to end the process, or may return to the WaveScope
process.  Thereafter, the process is ``broken'', and pending or future
calls to entrypoints will return a non-zero error code.

\section{Inline C Code}

The function for generating and including C code in the compiler's
output is {\tt inline\_C}.  We want this bso that we can
{\em generate} new/parameterized C code (by pasting strings together) rather than 
including a static {\tt .c} or {\tt .h} file, and instead of using
some other mechanism (such as the C preprocessor) to generate the C code.
The function  is called as ``{\tt inline\_C({\em c-code},
{\em init-function})}''.  Both of its arguments are strings.  The first
string contains raw C-code (top level declarations).  The second
argument is either the null string, or is the name of an
initialization function to add to the list of initializers called
before {\tt wsmain} is called (if present).  This method enables us to
generate, for example, an arbitrary number of C-functions dependent on
an arbitrary number of pieces of global state.  Accordingly we also
generate initializer functions for the global state, and register them
to be called at startup-time.

The return value of the {\tt inline\_C} function is a bit odd.  It
returns an empty stream (a stream that never fires).  This stream may
be of any type; just as the empty list may serve as a list of any
type.  This convention is an artifact of the WaveScript metaprogram
evaluation.  The end result of metaprogram evaluation is a dataflow
graph.  For the inline C code to be included in the final output of
the compiler, it must be included in this dataflow graph.  Thus {\tt inline\_C}
returns a ``stream'', that must in turn be included in the dataflow
graph for the inline C code to be included.
%
You can do this by using the {\tt merge} primitive to combine it with any
other Stream (this will not affect that other stream, as {\tt
inline\_C} never produces any tuples).  Alternatively, you can return the
output of {\tt inline\_C} directly to the {\tt BASE} (the root sink),
as follows:

\begin{wscode}
BASE <- inline\_C(\dots)
\end{wscode}

Note that a WaveScript program can contain multiple ``{\tt BASE <-}''
statements that are implicitly combined with the {\tt merge}
primitive.  Of course, all the streams returned in this way must be of
the same type.
%
\rednote{Actually, this feature is planned but not implemented at the
  moment [2007.08.24].}




\section{Backend support}

The foreign interface works to varying degrees under the {\bf Scheme},
 {\bf MLton}, and {\bf C++/XStream} backends ({\tt ws}, {\tt wsmlton},
 and {\tt wsc} respectively).  Below we discuss the current  limitations in each
 backend.  The feature matrix in Figure \ref{f:features} gives an
 overview.

\begin{figure}
\begin{center}
\begin{tabular}{|r|r|r|l|}
\hline $feature$ & $ws$ & $wsmlton$ & $wsc$ \\
\hline

{\tt foreign}         & yes     & yes & yes \\
{\tt foreign\_source} & never   & yes & not yet \\
{\tt inline\_C}       & not yet & yes & not yet \\

loads {\tt .c}        & yes     & yes     & yes \\
loads {\tt .h}        & yes     & yes     & yes \\
loads {\tt .o}        & yes     & not yet & yes \\
loads {\tt .a}        & yes     & not yet & yes \\
loads {\tt .so}       & yes     & no      & yes \\


marshal scalars    & yes     & yes      & yes \\
marshal arrays     & no      & yes      & not yet \\
{\tt ptrToArray}   & no      & yes      & not yet \\
{\tt exclusivePtr} & yes     & not yet  & yes \\

\hline
\end{tabular}
\end{center}
\caption{Feature matrix for foreign interface in different backends}
\label{f:features}
\end{figure}

Note that even though the Scheme backend is listed as supporting {\tt
.a} and {\tt .o} files, the semantics are slightly different than for
the C and MLton backends.  The Scheme system can only load
shared-object files, thus when passed {\tt .o} or {\tt .a} files, it
simply invokes a shell command to convert them to shared-object files
before loading them.

Including source files also has a slightly different meaning between
the Scheme and the other backends.  Scheme will ignore header files
(it doesn't need them).  Then C source files ({\tt .c} or {\tt .cpp})
are compiled by invoking the system's C compiler.  On the other hand,
in the XStream backend, C source files are simply {\tt \#include}d
into the output C++ query file.  In the former case, the source is
compiled with no special link options or compiler flags, and in the
latter it is compiled under the same environment as the C++ query file
itself.  

Thus the C source code imported in this way must be fairly
robust to the {\tt gcc} configuration that it is called with.
%  If
%greater control is needed it is recommended to precompile the C code
%into a library or object file.
If the imported code requires any customization of
the build environment whatsoever, it is recommended to compile them
yourself and import the object files into WaveScript, rather than
importing the source files.

\rednote{[2007.05.03] Note: Currently the foreign function interface is only
supported on Linux platforms.  It also has very preliminary support
for Mac OS but has a long way to go.}




\section{Converting WaveScript and C types}

An important feature of the foreign interface is that it defines a set
of mappings between WaveScript types and native C types.  The compiler
then automatically converts, where necessary, the representation of arguments to foreign
functions.
This allows many C functions to be used without modification, or ``wrappers''.  Figure
\ref{f:types} shows the mapping between C types and WaveScript types.

\rednote{[2007.08.24] Currently wsmlton does not automatically null
  terminate strings.  This needs to be fixed, but in the meantime the
  user must null terminate them manually.}

\begin{figure}
\begin{center}
\begin{tabular}{|r|r|l|}
\hline
$WaveScript$ & $C$ & $explanation$\\
\hline
{\tt Int}   & {\tt int}   & 
  \parbox[t]{2.2in}{native ints have a system-dependent 
length, note that in the Scheme backend WaveScript {\tt Int}s may 
have less precision than C {\tt int}s} \\

{\tt Float} & {\tt float} & 
\parbox[t]{2.2in}{WaveScript floats are single-precision}\\

{\tt Double} & {\tt double} & \\

{\tt Bool} &   {\tt int} & \\

{\tt String} & {\tt char*} & pointer to null-terminated string \\

%()   & 

{\tt Char} & {\tt char} &  \\

{\tt Array T} & {\tt $T$*} & \parbox[t]{2.2in}{
pointer to C-style array of elements of type {\tt T}, where {\tt T}
must be a scalar type
}\\

{\tt Pointer} & {\tt void*} &  \parbox[t]{2.2in}{
 Type for handling C-pointers.  Only good for
  passing back to C.
}\\

\hline
\end{tabular}
\end{center}
\caption{Mapping between WaveScript and C types.  Conversions
  performed automatically.}
\label{f:types}
\end{figure}

\rednote{[2007.05.03] The system will very soon support conversion of
  Complex and Int16 types.  
%Further, it might provide a
%  C-library for manipulating the representations of other WaveScript
  types.
}


\section{Importing C-allocated Arrays}

A WaveScript array is generally a bit more involved than a C-style
array.  Namely, it includes a length field, and potentially other
metadata.  In some backends ({\tt wsc}, {\tt wsmlton}) it is easy to
pass WaveScript arrays to C without copying them, because the WS array
contains a C-style array within it, and that pointer may be passed
directly.

Going the other way is more difficult.  If an array has been allocated
(via {\tt malloc}) in C, it's not possible to use it directly in
WaveScript.  It lacks the necessary metadata and lives outside the
space managed by the garbage collector.  However, WaveScript does
offer a way to {\em unpack} a pointer to C array into a WaveScript
array.  Simple use the primitive {\tt ``ptrToArray''}. But, as with
foreign functions, be sure to include a type annotation.  (See the
table in Figure \ref{f:features} for a list of backends that currently
support {\tt ptrToArray}.)

\section{``Exclusive'' Pointers}

Unfortunately, {\tt ptrToArray} is not always sufficient for our
purposes.  When wrapping an external library for use in WaveScript, it
is desirable to use memory allocated outside WaveScript, while
maintaining a WaveScript-like API.  For instance, consider a Matrix
library based on the Gnu Scientific Library (GSL), as will described in the
next chapter.  GSL matrices must be allocated outside of WaveScript.
Yet we wish to provide a wrapper to the GSL matrix operations that
feels natural within WaveScript.  In particular, the user should not
need to manually deallocate the storage used for matrices.

For this purpose, WaveScript supports the concept of an {\em
  exclusive} pointer.  ``Exclusive'' means that no code outside of
WaveScript holds onto the pointer.  Thus when WaveScript is done
with the pointer the garbage collector may invoke {\tt free} to
deallocate the referenced memory.  (This is equivalent to calling {\tt
  free} from C, and will not, for example, successfully deallocate a
  pointer to a pointer.)

Using exclusive pointers is easy.  There is one function {\tt
  exclusivePtr} that converts a normal {\tt Pointer} type (machine
  address) into a managed exclusive pointer.  By calling this, the
  user guarantees that that copy of the pointer is the only in
  existence.  Converting to an exclusive pointer should be thought of
  as ``destroying'' the pointer---it cannot be used afterwards.  To
  retrieve a normal pointer from the exclusive pointer, use the {\tt
  getPtr} function.  \rednote{In the future, getting an exclusive
  pointer will ``lock'' it, and you'll have to release it to make it
  viable for garbage collection again.  Currently, this mechanism is
  unimplemented, so you must be careful that you use the {\tt Pointer}
  value that you get back immediately, and that there is a live copy
  of the exclusive pointer in existence to keep it from being free'd
  before you finish with it.}
