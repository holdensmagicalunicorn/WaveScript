
%\documentstyle[refman.sty]{article}
\documentstyle{report}

% page size commands

\title{WaveScript 0.1 Users Manual}

\begin{document}

\maketitle

% title page commads

\pagenumbering{arabic}
\tableofcontents
\clearpage

\chapter{Introduction}

\~/wavescript

\section{Using WaveScript}

See the README for install instructions.  Once the system is installed
you should be able to execute {\tt ws} or {\tt wsc} on any WaveScript
source files ({\tt .ws}).

\subsection*{Develop Incrementally}

WaveScript is still very much a prototype.  It doesn't offer the same
level of support that you can expect from production-quality
compilers.  For example, the compiler doesn't track code locations.
It will often print the surrounding context of an error (in abstract
syntax), but nothing more.

Thus it helps to build your program bit by bit.  Compile often to make
sure that the pieces of your program type-check.

{\tt \bf wsparse:}
If you want the full error message for a stubborn parser error, run
{\tt wsparse} directly on the ws file.  It will either print the AST
(abstract syntax tree) for the file, or will give you an error message
that includes line and character numbers.


\section{Syntax}

Please make liberal use of ~/wavescript/demos/wavescope

\section{Naming conventions}


\chapter{Basics}

\section{Tuples}

\section{Numbers}

% int, uint16, uint8, float, double, complex 

% Could just have NUMBER in the FIRST type check..
% Then type check again AFTER elaboration.
% (It would be really nice to keep source info....)

\section{Streams}

\section{Sigsegs}

\section{Functions}

\subsection*{Patterns}

In any variable-binding position it is valid to use a pattern rather
than a variable name.  This includes the arguments to a function, a
local variable binding, or the variable binding within an {\tt
  iterate} construct.  Currently, patterns are just used to bind names
to the interior parts of tuples.  In the future, we will support list
patterns, array patterns, and tagged union patterns.  Let's look at an example.

We saw above how to bind variables in WaveScript:
\begin{center}
{\tt \bf{z} = (1,2);}
\end{center}
This binds {\tt t} to a tuple containing two elements.  
Now I will share with you that this is actually a shorthand for the
more verbose syntax:
\begin{center}
{\tt let \bf{z} = (1,2);}
\end{center}
Now we are ready to bind the individual components of the tuple by using
a pattern in place of the variable {\tt 'z'}:
\begin{center}
{\tt let \bf{(x,y)} = (1,2);}
\end{center}
Note: an unfortunate limitation of the parser is that {\tt 'let'} cannot be
omitted if we a pattern is used in place of a simple identifier.

Similarly, we may use patterns within a function's arguments.  Here's
a function that takes a 2-tuple as its second argument:

\begin{center}
{\tt {\bf fun} foo (x,\bf{(y,z)}) \{ \dots \}}
\end{center}


\chapter{Signal Processing Libraries}



\chapter{WaveScript Evaluation Model}




\end{document}
