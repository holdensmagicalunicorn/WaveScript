
%\documentstyle[refman.sty]{article}
%\documentstyle[twocolumn]{report}
\documentclass[twocolumn]{report}
\usepackage{color}
\usepackage{code}

%\usepackage{hyperref}
\usepackage{html}


\setlength{\textheight}{9.25in}
\setlength{\columnsep}{0.33in} %RRN changed: {0.33in}
\setlength{\textwidth}{7.48in}
\setlength{\footskip}{0.0in}
\setlength{\topmargin}{-0.25in}
\setlength{\headheight}{0.0in}
\setlength{\headsep}{0.0in}
\setlength{\oddsidemargin}{-.45in}
\setlength{\parindent}{1pc}


% page size commands

%% Need to automatically grab the revision number.
\title{WaveScript Rev 3493 User Manual}

\newcommand{\rednote}[1]{{\textcolor{blue}{#1}}}
%\newcommand{\rednote}[1]{{\em \bf{}}}

\newcommand{\evalsto}[2]{\[ \begin{array}{rcl}
$\tt #1 $&$ \arr $&$ #2 $
\end{array} \]}

\newcommand{\arr}{\ensuremath{\rightarrow}}

\newcommand{\cde}{\tt}

\newcommand{\ws}{WaveScript}

\newcommand{\ra}{\ensuremath{\rightarrow}}

\newenvironment{wscode}{\begin{center}\tt}{\end{center}}

\begin{document}

\maketitle

% title page commads

\pagenumbering{arabic}
\tableofcontents
\clearpage

\chapter{Introduction}

%\rednote{NOTES FOR THIS PARTICULAR REVISION:}

% [2007.12.01] nixing metaprogramming ...
WaveScript is a functional language for writing highly abstract
programs that generate efficient dataflow graphs through a two-stage
evaluation model.  These graphs are executable in several backends
(Scheme, ML, C, Java, TinyOS) that offer different tradeoffs in
features, performance, and compile times, as well as parallel and
distributed execution.  The compiler has several command line
entrypoints for the different backends.

%Calling {\bf ws}
% linking them to the {\em XStream} engine.

%\section{Bottom-up Tutorial}
\section{A taste of the language}\label{s:taste}

The academic publications on WaveScript will give you a top-down
account of its features, reason for being, and so on.  In this manual,
our goal is instead to introduce you to programming in WaveScript in a
bottom-up manner.  We'll start from ``Hello-world'' and proceed from
there.

%Here's the simplest complete WaveScript program that you can write:
Here's a simple but complete {\ws} program:
\begin{verbatim}
  main = timer(10.0)
\end{verbatim}
This creates a timer that fires at 10hz.  The {\em return value} of
the {\cde timer} function is a stream of empty-tuples (events carrying
no information).  The return value of the whole program is, by
convention, the stream named ``main''.  The {\em type} of the above
program is {\cde Stream ()}, where {\cde ()} designates the empty-tuple.

In our next program, we will print ``Hello world'' forever.
\begin{verbatim}
  main = iterate x in timer(10.0) {
           emit "Hello world!";
         }
\end{verbatim}
The {\cde iterate} keyword provides a special syntax for accessing
every element in a stream, running arbitrary code using that element's
value, and producing a new stream as output.  In the above example,
our {\cde iterate} ignores the values in the timer stream (``x''), and
produces one string value on the output stream on each invocation
(thus, it prints ``Hello world!'' at 10hz).  The type of the program
is {\tt Stream String}. 

Note, that we may produce two or more elements on the output stream
during each invocation.  For example, the following would produce two
string elements for each input element on the timer stream.
\begin{verbatim}
  main = iterate x in timer(10.0) {
           emit "Hello world!";
           emit "Hello again!";
         }
\end{verbatim}

Actually, timers are the only built-in {\ws} stream {\em sources}.
All other built-in stream procedures only transform existing streams.

Iterate also allows persistent, mutable state to be kept between
invocations.  This is introduced with the sub-keyword {\cde state}.
For example, the following produces an infinite stream of integers
counting up from zero.

\begin{verbatim}
  main = iterate x in timer(10.0) {
           state { cnt = 0; }
           emit cnt;
           cnt := cnt + 1;
         }
\end{verbatim}

Notice that the assignment operator for mutating mutable variables
({\cde :=}) is different than the operator used for declaring new
variables ({\cde =}).  ({\ws} also has {\cde +=}, {\cde -=} etc for use
in mutating variables.)  It is also possible for mutable state to be
declared {\em outside} the scope of the {\cde iterate}, but that introduces
the possibility of referring to the state from the bodies of two
different {\cd iterates}, which is not allowed.

As a final example, we'll merge two streams operating at different
frequencies.

\begin{verbatim}
  s1 = iterate x in timer(3.0) { emit 0; }
  s2 = iterate x in timer(4.0) { emit 1; }
  main = merge(s1,s2)
\end{verbatim}

This will output a sequence of zeros and ones, with four ones for
every three zeroes.  The {\cde merge} operator combines two streams of
the same type, interleaving their elements in real time.

Before we get into the nitty gritty, 
I will leave you with two high-level points regarding how you should
think of WaveScript.

\begin{center}
\begin{enumerate}
   \item {\bf Everything is a value.}  Streams and functions are just
   values (or objects if your prefer).  The purpose of a WaveScript
   program is to return a Stream.

   \item {\bf But you need to understand multi-stage evaluation.}
   There are some restrictions on the code that executes inside {\tt
   iterate}s, having to do with what features are supported at runtime
   vs. compile time.  This detracts from the simplicity of (1),
   because, for example, function values may not be stored at run time.
   We'll return to this issue in Chapter \ref{s:evalmodel}.
\end{enumerate}
\end{center}


\section{Pragmatics: Using WaveScript}

% This section discusses to setup and run {\ws}, what the different
% backends are, and how to make best use of them.  If you want to first
% find out more about the language, skip to Chapter~\ref{s:lang} and
% return here later.

This section addresses pragmatic issues concerning invoking
the system and running dataflow graphs.  If you are new to WaveScript,
it may help to skip to chapter \ref{s:lang} to get a sense for the
language itself, then come back to this section afterwards.

\subsection*{Installing}

See the README for install instructions.  The bottom line is that you
will need Mzscheme and Petite Chez Scheme to run the WaveScript
compiler and the Scheme backend.  (Petite for Linux is included in the
repository.)  You will need Boost, the Gnu Scientific Library (GSL),
FFTW, and their respective headers to run compiled C++ code.

Once the system is installed you should be able to execute {\bf \tt
  ws} on any WaveScript source files ({\tt .ws} files).  This will
compile the query and execute it directly in the Scheme backend.
Other backends (and variations on this one) will be discussed in this
section.  

\subsection*{Develop Incrementally}

WaveScript is a research compiler.  It doesn't offer the same
level of support that you expect from production-quality
compilers.  Sometimes the error messages might be difficult to
interpret.  (However, the type checker messages have been improved
immensely and do include code locations [04/08/2007].)

Therefore it helps to build your program bit by bit.  Compile often to make
sure that the pieces of your program parse and type-check.

{\tt \bf wsparse:} While entrypoints such as {\tt \bf ws} parse source files
internally, if you want the full error message for a parser error
(including line and column number!), run {\tt wsparse} directly on the
{\tt .ws} file.  It will either print the AST (abstract syntax tree) for the
file, or will give you an error message that does include line and
character numbers.

\subsection{{\tt \bf ws}: The Interactive Scheme Backend}

Running the command ``{\tt ws mysource.ws}'' will compile your source
file, and execute the resultant dataflow graph directly within Scheme,
without generating code for another platform.
%
The dataflow graph is converted to a Scheme program which is compiled
to native code and executed on the fly.  Compile time is low, because
we need not call a separate backend compiler.  However, performance is
not as good as the other backends we will discuss.  Further, the
Scheme backend does not support external (real time) data sources,
only queries running in virtual time using data stored in files.

One major advantage of the Scheme backend is that it's {\em
  interactive}.  The query starts up paused, and the user may step
through and inspect the elements of the output stream.  Thus the user
may {\em pull} on the stream rather than being {\em pushed} a torrent
of output.

%% Further, hardware
%% sensors cannot be accessed from the emulator (except through the
%% foreign interface \ref{s:foreign}), and so data must be read from
%% trace files.

\subsection*{Variants of {\tt \bf ws}}

There are several other useful ways to invoke WaveScript with the
Scheme backend.

\subsubsection*{Consider {\tt \bf ws.opt} for higher performance}

The first thing to do to improve the performance of {\tt ws} is make
sure that you have the full version of Chez Scheme.  This version
includes the incremental native code compiler.  If you are using the
free, Petite Chez Scheme interpreter, your {\tt .ws} query files will
be converted to {\em interpreted} Scheme code, which runs considerably
slower.  The executable {\tt chez} should be in your path.

To go further, you might consider running {\tt ws.opt} in place of {\tt
ws}.  First, {\tt ws.opt} has debugging assertions disabled. 
Second, {\tt ws.opt} is compiled in the highest Chez Scheme optimize
level, and also compiles your query in this higher optimize level.
However, this optimize level is dangerous in that it omits various
checks necessary to maintaining type safety.  Thus the process can
crash if there's an error, or memory can become corrupted (just like
C/C++).  Therefore, debug your code using {\tt ws} first.

\subsubsection*{Run {\tt \bf ws.debug} occasionally for sanity-checking}

Ws.debug is too slow for the normal development cycle, but it is
important to occasionally compile your program with ws.debug.  Mainly
this is important for building confidence that the compiler is
behaving properly on your program.  If the
compiler were a mature system not under active development, this would
not be as important.  


Specifically, when you run {\tt ws.debug}:

\begin{enumerate}
\item It turns on extra ASSERT statements in the code that check data
structure invariants to help ensure proper operation.

\item It dumps the whole program at several points in the
  compilation to files such as ``.\_\_elaboratedprog.ss''.  It can be
  helpful to look at these for debugging.

\item It includes additional type checks of the entire program between
  passes (regular {\tt ws} type checks several times, but {\tt
  ws.debug} does more).  This helps expose compiler bugs.

\item Finally, it does ``grammar checks'' on the output of each compiler
  pass. Each pass includes a BNF grammar that describes its output.
  {\tt ws.debug} mechanically checks to make sure every intermediate
  expression conforms to the appropriate grammar.  Again, this helps
  to expose compiler bugs.
\end{enumerate}


\subsubsection*{For very low compile times: {\tt \bf ws.early}}

If you are running many small programs --- stream ``queries'' --- then
the latency of compilation may be more important to you than the
throughput of the query once it is running.  In this case it is
reasonable to use {\tt ws.early}.


%When you run {\bf ws.debug} the compiler runs extensive
%invariant checks on the data structures it uses, in particular it
%checks the intermediate forms between compiler passes for
%conformance to explicit grammars.  Further it inserts extra
%type-checks to make sure that the intermediate programs are well-formed.

\section{Premier Compiler Backends}

\subsection{The WaveScript MLton Backend}

The WaveScript Scheme backend (ws) provides a reference implementation suitable
to prototyping and debugging applications.  The eventual target for
most WaveScript programs is to generate an efficient stand-alone
binary using one of our other backends.  One of these backends is the
MLton backend, invoked with {\bf wsmlton}.

MLton is an aggressive, whole-program optimizing compiler for Standard
ML.  Generating ML code from the {\ws} dataflow graph is
straightforward because of the similarities between the languages'
type systems.  This provides us with an easy to implement
single-threaded solution that exhibits surprisingly good performance
\cite{mlton}, while also ensuring type- and memory-safe
execution.  




\subsection{The ``Low-Level'' C Backend}

Invoked with ``{\bf wsc2}''.

Moving forward, this will serve as the {\em primary} WaveScript
backend.  It has three purposes.  First, it generates the most
efficient code.  Second, it serves as a platform for us to build
custom garbage collectors (the C++ backend simply uses Boost smart
pointers) and experiment with other aspects of the runtime.  Third, it
generates code with minimal dependencies (not requiring a runtime
scheduler).  This last point was critical in adapting the backend to
target TinyOS (described in the next section).

Currently [2008.08.15], {\bf wsc2} includes several garbage collection
modes (simple reference counting, specialized deferred reference
counting, boehm), has some support for multithreaded execution, some
support for client/server execution over a LAN using ssh, and some
support for client/server execution over a network of sensor nodes
running JavaME or TinyOS and a base station.  (See the following
section.)

\subsection{The NesC/TinyOS 2.0 Backend}

Invoked with ``{\bf wstiny}''.

This backend ``inherits'' from {\bf wsc2} and modifies it to support
TinyOS.  As of this writing [2008.02.22], the TinyOS backend is
limitted in its supported language features.  (See the end of this
section for a list of limitations.)

{\bf wstiny} can be invoked in multiple ways.  First, simply invoking
{\bf wstiny} on a WaveScript file (in the same way as any other backend)
will compile the entire application to run on the mote.

\begin{verbatim}
  [joe@computer] wstiny demo11g_tinyos.ws
\end{verbatim}

In the same way that one invokes ``{\tt query.exe}'' or ``{\tt
  query.mlton.exe}'' when using another WaveScript backend, one can
run the {\bf wstiny} program with ``{\tt query.py}'':

\begin{verbatim}
  [joe@computer] ./query.py
\end{verbatim}

This is a python script that invokes TOSSIM, the TinyOS simulator.
The output of the program will be printed to the terminal---including
printed output and return values on the main stream---again, just like
any other backend.

Of course, the point of compiling for TinyOS is to run code on a {\em
  mote}!  This is slightly more involved.  After executing {\bf
  wstiny}, we need to build the binary for the particular mote
platform.  And then we invoke a listener on the serial port to see the
output of the program.

\begin{verbatim}
[joe@computer] make -f Makefile.tos2 telosb install
[joe@computer] MOTECOM=serial@/dev/ttyUSB0:telosb
[joe@computer] export MOTECOM
[joe@computer] java PrintfClient 
\end{verbatim}

%The ``-comm'' argument can be omitted if the MOTECOM environment
%variable is set (see the TinyOS documentation).  

{\bf wstiny} has polluted your working directory with several files,
including  {\tt Makefile.tos2}, {\tt PrintfClient.java}, {\tt query.py} and others.
%
The PrintfClient program can listen to a particular device specified
either through the {\tt MOTECOM} environment variable, or through a
``{\tt -comm}'' flag.  Please refer to the the TinyOS documentation
for instructions on how to connect motes to your computer.

\subsubsection*{Executing across mote {\em and} PC}

Alas, it is not very interesting to simply run code on a mote that is
attached to a PC's serial port.  For a real sensor network
application, we need to run WS code on an entire sensor network {\em
  and} on the PC connected to the network, seemlessly splitting the
application across the two tiers.

Currently, {\bf wstiny} supports a method for manually splitting
programs across tiers.  This is implemented using a simple naming
convention.  A ``{\cde Node}'' namespace contains all the streams that will
live inside the network, and everything outside of that namespace
lives on the PC.  The following program illustrates this method, but
it may help to return to it after reading Section~\ref{s:lang}, and in particular
Section~\ref{s:namespaces} which explains namespaces.

\begin{verbatim}
  namespace Node {
    s1 = ...
    s2 = ...
  }
  main = smap((+ 1), Node:s2);
\end{verbatim}

Only the ``(+ 1)'' operation is executed on the PC.  The ``cut point''
that splits {\cde s2} between the mote and PC will cause the compiler
to generate code on both sides that accomplishes the communication.
To compile this program, and execute it across the PC and a single
mote connected to a serial interface, we type the following:

\begin{verbatim}
[joe@computer] wstiny -split demo11g_tinyos.ws
[joe@computer] make -f Makefile.tos2 telosb install
[joe@computer] ./query.exe
\end{verbatim}

First, we compile the program, producing code for both platforms.
Second, we install the progam onto the mote.  Third, we execute the PC
portion of the program, which listens to the serial port, unpacks the
messages from the stream {\cde s2}, and executes the rest of the
program, producing output to {\tt stdout} as usual.
%
If we had compiled {\em without} {\cde -split}, then the entire
application would run on the mote.

Note, the mote-side communication is accomplished using the TinyOS
interface {\cde AMSend}, which by default is wired to a {\cde
  SerialAMSenderC} component.  If we want to execute on more than one
mote, we must instead wire this interface to the radio (the {\cde
  AMSenderC} component).  This is accomplished with:

\begin{verbatim}
[joe@computer] wstiny -split-radio demo11g...
\end{verbatim}

Which is just shorthand for:

\begin{verbatim}
[joe@computer] wstiny -split-with "AMSenderC" ...
\end{verbatim}

The resulting NessC code can be used to program several motes.  Then
the TinyOS-included ``BaseStation'' application can be used to enable
a PC to receive these radio messages on its serial port.  This method
described in the here:  \htmladdnormallink{Lesson 4: Mote-PC serial communication}
  {http://www.tinyos.net/tinyos-2.x/doc/html/tutorial/lesson4.html}.

Note that the semantics of this method is that the messages from {\em
  every} mote's {\cde s2} stream are combined to form the {\cde s2}
stream seen on the PC.  Thus, the {\cde s2} streams will most likely
include the a node identifier so that the PC can demultiplex the
stream.  Please see Section~\ref{s:foreigntos} for information on integrating
legacy TinyOS code with a WaveScript application.

\subsubsection*{Limitations}

\begin{itemize}
\item The above {\tt -split} methodology only works for
  one-hop networks.  We must adapt it to use the CTP (Collection
  Tree Protocol) to work over larger networks.
\item Code running on motes may not currently allocate memory
  dynamically.
\item Only string constants, static arrays, numbers, and tuples are
  currently supported.
\item Not all datatypes can be sent across a ``cut'' stream.
  Currently, tuples and numbers support marshaling (including nested
  tuples).  Arrays support marshaling but *only* if the array is not
  contained within a tuple.
\end{itemize}

\subsubsection*{Future plans}

We are in the process of removing the above limitations and adding
support for new features:

\begin{itemize}
\item Dynamic allocation and garbage collection.
\item Automatic and semi-automatic partitioning of stream programs
  across PCs and motes.
\end{itemize}




\section{Deprecated Compiler Backends}

\subsection{The WaveScript CaML Backend}

\rednote{This backend is no longer supported and will receive no new
  features!  But if you need to use it, let me know.}

\subsection{The C++/XStream Backend}

\rednote{This backend is no longer supported and will receive no new
  features!  But if you need to use it, let me know.}

The C++ backend\footnote{It uses a C++ compiler not because it
  generates object-oriented code, but because the XStream runtime engine it
  links with has been engineered in C++.} generates code that uses the
XStream runtime/scheduling engine.  It is invoked with the {\bf wsc}
command.  
In practice, as WaveScript undergoes development, the {\bf wsc}
compiler often lags behind {\bf ws} in terms of features and
functionality.  Again, develop incrementally, refer to {\tt
  demos/wavescope} for programs that should work with {\bf
  wsc}.

Refer to the XStream documentation\footnote{\rednote{[2007.03.09] Currently nonexistent}} 
for information on how to configure the XStream engine
(number of threads, scheduling policy, etc)).

\rednote{[2007.03.09] Ideally the output from the XStream-executed
query would be the same as the ws's output.  Currently, however, there are
some basic disparities in how objects are printed to text form.  Hopefully, these
are straightforward to work around, and should go away.}

Note that the C++/XStream backend does not have competitive performance with
either the MLton backend or the newer ANSI C backend ({\bf wsc2}).


%================================================================================
\chapter{Language Basics}\label{s:lang}

Please make liberal use of the demo files located at {\tt demos/wavescope/*.ws} as a reference
when learning WaveScript.  These demos cover most of the basic
functionality in the language.

%\section{Naming conventions}
%\chapter{Basics}

Wavescript is a functional language with many similarities to ML or
Haskell.  Functions are values.  The language is strongly typed and uses a
type-inference algorithm to infer types from variable usages.
It provides the usual basic datatypes, including various
numeric types, strings, lists, arrays, hash tables as well as the more
WaveScript-specific Streams and Sigsegs.  Valid {\em expressions} in
WaveScript are written much as in C.
\[
\begin{array}{rcl}
arithmetic       & : & $\tt 3 + 4 * 10 $  \\
function\; calls & : & $\tt f(x, y)$  \\
blocks           & : & $\tt \{ e$_1$; e$_2$; \ldots$\:$ e$_n$ \}$  \\
\end{array}
\]

But there are also many syntactic differences from C.  For example,
WaveScript doesn't have a sharp division between commands and
expressions.  Conditionals use a different syntax and are valid in
{\em any}
expression position.
%
\evalsto{3 + (if true then 1 else 2)} {4}
%
(Note that the arrow above means ``evaluates to'', but {\ws} does not
currently have an interactive read-eval-print-loop in which you can
type such incomplete program fragments.)

Moreover, {\em blocks} in WaveScript---delimited by curly
braces---are just expressions!

\evalsto{3 + \{1; 2; 3\}} {6}
%
This is similar to the {\tt begin}/{\tt end} blocks found in other
functional languages, or the ``comma operator'' sequencing construct
in C++.  Only the value of the last expression within the block is
returned, the other statements/expressions only bind variables or
perform side effects.

Note that this different convention makes semi-colon usage in
WaveScript somewhat unintuitive.  Within blocks, semi-colons are only
required as {\em separators}; they are permitted, but not required
after the last expression in the block.  Also, since curly-brace
delimited blocks are merely expressions, they sometimes must be
followed with a semi-colon, as below:\footnote{As a special case, when
a function body consists of curly braces, it needn't be followed by a semi-colon.}

%\vspace{-5mm}
\begin{verbatim}
        {
          foo;
          if b then {
            bar();
          } else {
            baz();
          };  <-- SEMICOLON REQUIRED
          done();
        }
\end{verbatim}

\rednote{[2007.03.09] Be wary that with the current parser, a
  semi-colon error may appear as a strange parse error in the next
  or previous line.}


%================================================================================
\subsection{Streams}

WaveScript is for stream-processing and it would be useless without
Streams.  
Streams in WaveScript are like any other values: bound to variables,
passed and returned from functions.
The primary means of operating on the data inside streams is the {\tt
  iterate} syntactic construct.

\begin{verbatim}
        S2 = iterate x in S1  {
          state{ counter = 0 }
          counter := counter + 1;
          emit counter + x;
        }
\end{verbatim}

The {\tt iterate} construct can be placed in any {\em expression
position}.  It produces a new stream by applying the supplied code to
every element of its input stream.  The above example executes the
body of the iterate every time data is received on the stream {\tt
  S1}, it adds an increasing quantity to each element of {\tt S1}, and the
resulting stream is bound to the variable {\tt S2}.

In addition to {\tt iterate} several library procedures (such as {\tt
stream\_map}) and many primitives (such as {\tt unionList}) operate on
{\tt Stream} values.  For example, {\tt merge} combines two streams of the same
type (their tuples interleaved in real time), whereas {\tt
  unionList} also combines streams of the same type, but tags all
output tuples with an integer index indicating from which of the input
streams it originated.


%========================================
\subsection{Defining functions}
\label{udfs}

Named functions can be defined at the top-level, or within a {\tt
  \{\ldots\}} block, as follows:
\begin{verbatim}
       fun f(x,y) {
         z = x+y;
         sqrt(z)
       }
\end{verbatim}

The function body, following the argument list can be any expression.
Above it happens to be a statement block that returns the
square-root of {\tt z}.

Unnamed, or anonymous, functions can occur anywhere that an expression
might occur. In the following we pass an anonymous ``plus three''
function as the first input to {\tt stream\_map}.  Note that the body
of the function is a single expression, not a statement block, and is
not delimited by curly braces.

\begin{wscode}
stream\_map(fun(x) x+3, S)
\end{wscode}

%========================================
\subsection{Constructing complete programs}

A complete program file contains function and variable declarations.
As seen in Section~\ref{s:taste}, a special variable declaration
binds the name ``main'' to a stream:
%as well as a {\em stream wiring statement} of the form:
\begin{wscode}
main = stream-valued-expression;
\end{wscode}
%This wiring statement determines the stream of values returned as
%a result of running the program.
As a side-note, {\ws} can be called with a ``-ret foo'' flag to return
the stream ``foo'' instead of the stream ``main''.  This is frequently
useful for invoking testing entrypoints as well as the production
entrypoint for the program.  

Note also that {\em only} the returned stream is instantiated at runtime.
Other streams declared in the program will simply be ignored.  They
become dead code.


%================================================================================

\subsection{Tips for C/Java Programmers}

As a quick cheat sheet, refer to Figure \ref{f:cprogrammertips}.  This
addresses some minor syntactic differences between WS and C/C++ that
commonly trip people up.

One important piece of advice is to be careful with mutation ({\tt
  :=}).  Don't use it when it's unnecessary!  And if you need to use
  it, keep the scope of the variables in question as small as
  possible.  If, for example, you introduce mutable top-level (global)
  variables it may have major deleterious effects.  First of all, it
  opens up the possibility that you will refer to the state from
  multiple {\tt iterate} bodies, which will make the compiler yell at
  you.  More subtly, it can disable optimizations that act on {\tt
  iterate}s without persistent state---for example, replicating them
  in parallel.

\begin{figure}
\begin{center}
\begin{tabular}{|c|c|}
\hline {\em C/C++ code} & {\em WS Equivalent} \\
\hline
{\tt int x = 3;} & {\tt x = 3;     \ \ {\bf \em OR}}   \\
{\tt } & {\tt x = (3::Int);}  \\
{\tt } & {\tt x::Int = 3;}    \\
\hline
{\tt x = 4;}            & {\tt x := 4;} \\ 
\hline
{\tt if (a) b; else c;} & {\tt if a then b else c;} \\  \hline
{\tt typedef char myty;} & {\tt type myty = Char; } \\  \hline

{\tt List< Array<int> >}          & {\tt List (Array Int) } \\  \hline
\end{tabular}
\end{center}
\caption{Some highlighted syntactic differences between WaveScript and C/C++/Java.}
\label{f:cprogrammertips}
\end{figure}


%% \subsection{Variables, Mutation, Shared State}

%% When transitioning from C to WaveScript one touchy point is the
%% distinction between 

%% mutable
%% (modifiable) variables.  WaveScript, like ML and Scheme, allows
%% mutation, but programmers in these languages tend to deemphasize it.

%% As a reminder, a normal variable binding such as {\tt v = expression;}
%% introduces a new variable and must occur at top-level or inside a
%% {\{...\}} block.  The variable will be in scope after its
%% introduction.  

%% Such a variable may be mutated by the {\tt :=} operator.

%% Such a variable is not mutable by default.  To construct a mutable
%% variable, one must create a ``Ref'', and modify its value with the
%% {\tt :=} operator, as in the following example:
%% \begin{verbatim}
%% {
%%    myvar :: Ref Int = ref(3);
%%    myvar := 4;
%%    myvar * 2;
%% }
%% \end{verbatim}
%% Notice that myvar has a different type---it is not an Int, but a Ref
%% Int.  You could think of a Ref like a one-element array.  Except
%% wherever the myvar variable occurs, it is dereferenced automatically.
%% (You don't have to {\tt deref}.)

%% One additional complication in this scheme is that the {\tt state
%%   \{\}} block of an iterate implicitly makes all of the bound state
%%   variables ``ref''s.

%% In the future, implicit uses of ``ref'' may be allowed, but the {\tt
%%   :=} operator will remain distinct from the {\tt =} operator.

%Refer to {\tt prim_defs.ss} for more array operations.









%================================================================================
\section{Datatypes}

This section goes over syntax and primitives for manipulating
built-in data structures.

\subsection{Tuples}

Tuples, or {\em product types}, are ordered sets of elements which can
each be of different types. 
\evalsto{(1,``hello'')}{A tuple of type {\tt (Int * String)}}
%\begin{wscode}
Tuples are accessed by a pattern-matching against them and binding
variable names to their constituent components.  This process is
described in section \ref{s:patterns}.

\subsection{Numbers}

WaveScript supports several different types of numbers.  \rednote{Currently,
this set includes Int, Int16, Uint16, Int64, Float, Double, Complex, but it will be extended
to include 8-bit integers, other unsigned integers, and complex-doubles.}
WaveScript includes generic numeric arithmetic operations ($+$, $-$,
$*$, $/$) that work for any numeric types, but must be used on two
numbers of the same type.  There are also type-specific numeric
operations that are not used frequently:

\[
\begin{array}{rcl}
%generic          & : & +,\; -,\; $*$,\; /,\; ^{\wedge}   \\
generic          & : & $\tt + - * / \^{}$   \\
Int              & : & $\tt +\_ -\_ *\_ /\_ \^{}\_ $ \\
Float            & : & $\tt +. -. *. /. \^{}. $ \\
Complex          & : & $\tt +: -: *: /: \^{}: $ \\
Int16            & : & $\tt +I16 -I16 *I16 /I16 \^{}I16 $   \\
Int64            & : & $\tt +I64 -I64 *I64 /I64 \^{}I64 $   \\
\end{array}
\]

Other numeric operations, such as {\tt abs} or {\tt sqrt}, follow the
naming convention {\tt absI} for integers, {\tt absF} for floats, {\tt
absC} for complex, and {\tt absI16} for 16-bit integers.
\rednote{Eventually, {\ws} will include a type-class facility which
  will simplify the treatment of numeric operations.}

% int, uint16, uint8, float, double, complex 

% Could just have NUMBER in the FIRST type check..
% Then type check again AFTER elaboration.
% (It would be really nice to keep source info....)

\subsection{Lists}

Lists can be written as constants and support the usual primitive
operations.  
\begin{verbatim}
      ls = [1, 2, 3];
      ls2 = head(ls):::ls;
      print(List:length(ls2)); // Prints '4'
      print(ls == []);         // Prints 'false'
\end{verbatim}

The {\tt :::} operator adds an element to the front of a list.  Also
use {\tt head}, {\tt tail}, {\tt List:reverse}, {\tt List:append} to operate on
lists.  

Other operations include, but are not limited to the following. {\tt
List:zip} combines two equal-length lists into a list of two-tuples.
{\tt List:map(f,ls)} returns a new list generated by applying the
function {\tt f} to each element of the input list.  {\tt
List:filter(pred,ls)} returns only those elements of the list that
satisfy {\tt pred}.  {\tt List:fold(op,init,ls)} reduces a list by
repeatedly applying {\tt op} to pairs of elements, for example, to sum
the elements of a list.  {\tt List:mapi(f,ls)} is a variant of map
that also passes the index of the element to the input function.  {\tt
List:foreach(f,ls)} applies a function to each element for side effect
only, not building a new list.  {\tt List:build(len, f)} builds a new
list, using a function (on index) to populate each position in the
list.

Many of these operators (map, filter, foreach, mapi, fold, build, etc) are used consistently
for different container types (lists, arrays, matrices, etc).

\subsection{Arrays}

Arrays, unlike lists, are mutable. 
Use a hash symbol to build a constant array rather than a list, for
example {{\tt \#[1,2,3]}}.
Use
{\em Array:make} to allocate new arrays.  The {\em Array:ref} and {\em Array:set}
primitive access arrays elements, but it's shorter to use the syntactic
sugar, {\tt arr[i]} to access arrays, and {\tt arr[i] := x;} to modify them.  Nested array references work as
expected, as do the {\tt +=} style shorthands, e.g. {\tt arr[i][j] += 3}.  
However, to use these shorthands with an arbitrary, non-variable
expression extra parentheses are required: {\tt (f(x))[i]}.

Many array operations are analogous to the list operations (map,
fold, etc).  See the documentation for additional primitives below.

\subsection{Additional primitive functions}

Appendix~\ref{s:primtable} contains a table of all currently supported
{\ws} primitives, together with their type signatures.

For more documentation on these primitives, please refer to this file
within your working copy:
\vspace{-2mm}
\begin{center}
{\tt{src/generic/\\compiler\_components/prim\_defs.ss}}
\end{center}

You can find the online documentation at:
\vspace{-2mm}
\begin{center}
\url{http://regiment.us/codedoc/html/generic/compiler\_components/prim\_defs.ss.html}
\end{center}

%\htmladdnormallink {my Web page} {http://www.astro.ku.dk/\~{}milvang/}

%\htmladdnormallink{\tt{wavescript/src/generic/\\compiler\_components/prim\_defs.ss}}{http://regiment.us/codedoc/html/generic/compiler_components/prim_defs.ss}

%\htmladdnormallink{wavescript/src/generic/compiler\_components/prim\_defs.ss}{http://regiment.us/codedoc/html/generic/compiler\_components/prim\_defs.ss.html}

This file contains type-signatures (and minimal documentation) for all built-in
wavescript primitives, many of which are not covered in this manual.
Within the online documentation linked above you should look at 
\htmladdnormallink{regiment-primitives}{http://regiment.us/codedoc/html/generic/compiler\_components/prim\_defs.ss.html\#regiment-primitives}
which is defined in terms of 
\htmladdnormallink{regiment-basic-primitives}      {http://regiment.us/codedoc/html/generic/compiler\_components/prim\_defs.ss.html\#regiment-basic-primitives},
\htmladdnormallink{regiment-distributed-primitives}{http://regiment.us/codedoc/html/generic/compiler\_components/prim\_defs.ss.html\#regiment-distributed-primitives},
\htmladdnormallink{wavescript-primitives}          {http://regiment.us/codedoc/html/generic/compiler\_components/prim\_defs.ss.html\#wavescript-primitives},
\htmladdnormallink{meta-only-primitives}           {http://regiment.us/codedoc/html/generic/compiler\_components/prim\_defs.ss.html\#meta-only-primitives},
\htmladdnormallink{higher-order-primitives}        {http://regiment.us/codedoc/html/generic/compiler\_components/prim\_defs.ss.html\#higher-order-primitives},
and
\htmladdnormallink{regiment-constants}             {http://regiment.us/codedoc/html/generic/compiler\_components/prim\_defs.ss.html\#regiment-constants}~.

Please also examine the library files found in the {\tt lib/}
sub-directory.  These files, for example {\tt ``stdlib.ws''} and {\tt
  ``matrix.ws''}, include library routines written in WaveScript.  As
a general design principle, it is best to implement as much as
possible in the language, while keeping the set of built-in primitives
relatively small.

For various historical reasons there are several primitives included
in the current primitive table that {\em should not} ultimately be
primitive.  These will eventually be removed and implemented instead
as library routines.  Likewise, there are certainly many additional
primitives that one might like to see incorporated as the language matures.


%================================================================================
\section{Type Annotations}

We have already seen several types in textual form within this manual:
{\tt Float}, {\tt (Int * String)} These are valid WaveScript types.
WaveScript also has compound types such as {\tt Sigseg (Array Int)}.

Similar to Haskell or ML, the user is permitted, but not generally
required to add explicit type annotations in the program.  For
example, we may annotate a function {$\tt f$} with its type by writing
the following.
%
\begin{verbatim}
        f :: (Type, Type) -> Type;
        fun f(x,y) { ... }
\end{verbatim}

Indeed, a type annotation may be attached to any expression with
the following syntax:
\begin{wscode}
(expression :: Type)
\end{wscode}

Further, type annotations may be added to variable declarations with:
\begin{wscode}
 var :: Type = expression;
\end{wscode}

In general, compound types may be built from other types in several
ways.  
\[
\begin{array}{rcl}
tuples           & : & $\tt (T1 * T2 * T3) $   \\
lists            & : & $\tt List T1 $ \\
arrays           & : & $\tt Array T1 $ \\
hashtables       & : & $\tt HashTable (T1,T2) $ \\
functions        & : & $\tt (T1, T2, T3) -> T4 $ \\
\end{array}
\]
Note that parentheses must be used when nesting type
constructors as in {\tt List (List T)}.

\rednote{[2007.03.09] In the future, the user will be able to create
  their own type definitions and type constructors, including
  tagged-union or {\em sum} types.}


\subsection{Reading data with readFile}

The one place where type annotations {\em are} currently mandated is
when importing data from a file.  This is done with the primitive {\tt
readFile}.  Readfile is not a stream {\em source}, but rather is driven by
another stream, reading a tuple from the file for every element on its
input stream.  This makes it more general.
%
\begin{verbatim}
       main = 
         (readFile("foo.txt", "", timer(10.0))
          :: Stream (Int16 * Float))
\end{verbatim}

The above complete program reads space-separated values from a text
file, one-tuple-per-line.  In this case, each tuple contains two
values: a 16 bit integer, and a floating point value.
The {\tt readFile} primitive may also be
used to read data from a file in blocks (Sigsegs), which is generally
more efficient.  All that is required is to specify a stream of
{\tt Sigseg} type, as follows.

\begin{verbatim}
       main =
         (readFile("foo.dat", 
                   "mode: binary  repeat: 3", 
                   timer(10.0))
          :: Stream (Sigseg Int16))
\end{verbatim}

Also note that the second argument to {\tt readFile} is an option
string.  This string is parsed at compile time (during meta-program
evaluation).  The string must contain a (space separated) list of
alternating option names and option values.  The following are the
available options, and their default values.

\begin{itemize}
\item {\bf mode:} one of 'text' or 'binary' (default {\bf 'text'})

\item {\bf repeats:} a number specifying whether to replay the file's
  contents when the end of file is reached.  Set to a non-negative
  integer to specify the number of repeats, or to {\bf -1} to repeat
  indefinitely.  (default {\bf 0})

\item {\bf rate:} the rate (in tuples per second) to play the back the
  data from the file.  For the emulator, this refers to virtual time,
  and is used only to maintain relative timing of different data
  streams.  Note that this is orthogonal to windowing; whether data is
  windowed or not, the rate will be interpreted in the same
  way. (default {\bf 1000}) 

\item {\bf offset:}
  The offset, in bytes, at which to start reading from the
  file. (default {\bf 0})

\item {\bf skipbytes:}
The number of bytes to skip between reading each tuple
  from the data file. (default {\bf 0})

\item {\bf window:} If the output stream is blocked into Sigsegs
  (windows), this parameter determines the size of each
  Sigseg. (default {\bf 1})

\end{itemize}


%%  and reads them at a virtual ``rate'' of
%% 44Khz (this is important for the relative timing of different
%% streams).  Finally, the last number is the replay argument.  If it's greater
%% than zero, {\tt dataFile} will replay the data that many times after
%% reading the file.  If the replay value is $-1$, then the data will be
%% replayed indefinitely.



%================================================================================
\subsection{Type Aliases}

Because types can grow large and complex, it is often helpful to
define aliases, or shorthands, similar to C/C++'s {\tt typedef}s.

\begin{center}
\begin{verbatim}
       type MyType        = List Int;
       type MyType2  t    = Stream (List t);
       type MyType3 (x)   = List (x);
       type MyType4 (x,y) = List (x * y);

       x :: MyType;
       x = [3];

       s1 :: MyType2 Int;
       ...
\end{verbatim}
\end{center}


%================================================================================
\section{Sigsegs}

Sigsegs are a flexible ADT for representing windows of samples on a stream.
Please refer to the CIDR'07 publication with the title
``The Case for a Signal-Oriented Data Stream Management System'' for
details.  Also check {\tt prim\_defs.ss} for the specific names and
type signatures of the Sigseg primitives.

% smap$((+3))
% smap(f$(x), S)
%%  ls.List:ref(i)
%%  ls.list:ref(i)

%% list:ref 
%% List:ref 

%% list|ref 

%% ls.List|ref(i)


%% namespace Matrix {
  
%% }

%% Matrix:make
%% Array:make

%% using Matrix;




%================================================================================
\section{Namespaces}\label{s:namespaces}

WaveScript, while not having a sophisticated module system, does
include a simple system for managing namespaces.

\begin{center}
\begin{verbatim}
      namespace Foo {
        x = ...;
        y = ...;
      }
      var = Foo:x + Foo:y;
      fun f() {
        using Foo;
        var = x + y;
      }
\end{verbatim}
\end{center}


%================================================================================
\section{Patterns}
\label{s:patterns}

WaveScript allows pattern matching in addition to plain variable-
In any variable-binding position it is valid to use a pattern rather
than a variable name---this includes the arguments to a function, a
local variable binding, or the variable binding within an {\tt
 iterate} construct.  Currently, patterns are used to bind names
to the interior parts of tuples.  In the future, we will support list
patterns, and tagged union patterns.  

Let's look at an example.
We saw how to bind variables in WaveScript:
\begin{center}
{\tt \bf{z} = (1,2);}
\end{center}
This binds {\tt z} to a tuple containing two elements.  
This is actually a shorthand for the more verbose syntax:
\begin{center}
{\tt let \bf{z} = (1,2);}
\end{center}

An unfortunate limitation of the parser is that {\tt 'let'} cannot be
omitted if we a pattern is used in place of a simple identifier.  The
following binds the individual components of the tuple by using a
pattern in place of the variable {\tt 'z'}:
\begin{center}
{\tt let \bf{(x,y)} = (1,2);}
\end{center}

Similarly, we may use patterns within a function's argument list.  Here's
a function that takes a 2-tuple as its second argument:

\begin{center}
{\tt {\bf fun} foo (x,\bf{(y,z)}) \{ \dots \}}
\end{center}


%================================================================================
\section{Syntactic Sugar}

Syntactic sugars are convenient shorthands that are implemented by the
parser and make programming in WaveScript more convenient (at the risk
of making reading code more difficult for the uninitiated).

%\noindent {\bf Dot syntax:}
\subsection{Dot syntax}
  For convenience, functions can be applied using an
alternative ``dot syntax''.  For example rather than taking the first
element of a list with ``{\cde head(ls)}'', we can write ``{\cde ls.head}''.
This generalizes to functions of more than one argument; only the
first argument is moved before the dot.  For example, 
\vspace{-2mm}
\begin{wscode}
List:ref(ls,i)
\end{wscode}
\vspace{-2mm}
 may be written as 
\vspace{-2mm}
\begin{wscode}
ls.List:ref(i)
\end{wscode}
\vspace{-2mm}
This is useful because many functions on data structures take the
data structure itself as their first argument.  Thus it is concise to
write the following:
\vspace{-2mm}
\begin{wscode}
ls.tail.tail.head
\end{wscode}
\vspace{-2mm}


\subsection{\$: ``Unary parentheses''}

The ``\$'' operator for procedure application is taken from Haskell
and sometimes called the ``unary parenthesis''.  Instead of {\cde
  f(g(x))}, we write ``{\cde f \$ g \$ x}''.
This is useful if you have a large expression spanning many lines to
which you want to apply a function:
\begin{verbatim}
   iterate x in strm { 
      (many lines) ... 
\end{verbatim}
We can apply a function {\cde myfunction} without scrolling down to
insert a close parenthesis:
\begin{verbatim}
   myfunction $
   iterate x in strm { 
      (many lines) ... 
\end{verbatim}
%$

\subsection{Stream Projectors}

WaveScript also includes a syntax for binding streams of tuples in a
 way that associates a projector function for each of
the tuples' fields.  For example:

\begin{wscode}
S as (a,b) = someStream;
\end{wscode}

Subsequently, ``{\tt S.(a)}'' or ``{\tt S.(a,b,a)}'' can be used to
project a new stream where each tuple represents an arrangement of
the fields within each tuple in {S}.  If used in conjunction with the
type-annotation syntax, note that the ``{\tt as}'' clause must go first:

\begin{wscode}
S as (a,b) :: Stream(Int * Float) = someStream;
\end{wscode}



\chapter{Foreign (C/C++) Interface}

The WaveScript compiler provides a facility for calling external
(foreign) functions written in C or C++.  The primary reasons for this
are two-fold.

\begin{enumerate}
\item We wish to reuse existing libraries without modification (e.g. Gnu
  Scientific Library, FFTW, etc).

\item We wish to add new stream sources and sinks --- for network communication, disk access and so
  on --- without modifying the WaveScript compiler itself.  Also, we
  frequently want to add support for new hardware data-sources
  (sensors).
  
%In the future we will have a {\tt foreign\_box} syntax similar to {\tt
%  foreign}, but allowing the user to write a WSSource class usable by
%  the WaveScope engine.}.
\end{enumerate}

There are three WaveScript primitives used to interface with foreign
code.  The {\tt foreign} primitive registers a single C function with
WaveScript.  Alternatively, {\tt foreign\_source} imports a stream of
values from foreign code.  It does this by providing a C function that
can be called to add a single tuple to the stream.  Thus we can call
from WaveScript into C and from C into WaveScript.  The third
primitive is {\tt inline\_C}.  It allows WaveScript to generate
arbitrary C code at compile time which is then linked into the final
stream query.  We can of course call into the C code we generate from
WaveScript (or it can call into us).
%Thus we can generate C code and then call into the code
%we generated (or have it call into us).




\section{Foreign functions}

The basic foreign function primitive is called as follows: ``{\tt
foreign({\em function-name}, {\em file-list})}''.  Like any other
primitive function, {\tt foreign} can be used anywhere within a
WaveScript program.  It returns a WaveScript function representing the
corresponding C function of the given name.  The only restriction is
that any call to the {\tt foreign} primitive {\em must} have a type
annotation.  The type annotation lets WaveScript type-check the
program, and tells the WaveScript compiler how to convert (if
necessary) WaveScript values into C-values when the foreign function
is called.  

The second argument is a list of {\em dependencies}---files that must
be compiled/linked into the query for the foreign function to be
available.
%
For example, the following would import a function ``foo'' from ``foo.c''.

\begin{wscode}
c\_foo :: Int -> Int = foreign("foo", ["foo.c"])
\end{wscode}

%Note that, as when reading data from a file, the compiler has no way
%to know the type of the external function, so an explicit type
%annotation is required.

Currently C-code can be loaded from source files ({\tt .c}, {\tt
  .cpp}) or object files ({\tt .o}, {\tt .a}, {\tt .so}).  When
  loading from object files, it's necessary to also include a header
  ({\tt .h}, {\tt .hpp}).  For example:

\begin{verbatim}
c_bar = 
   (foreign("bar", ["bar.h", "bar.a"]) 
    :: Int -> Int)
\end{verbatim}

Of course, you may want to import many functions from the same file or
library.  WaveScript uses a very simple rule.  If a file has already
been imported once, repeated imports are suppressed.  (This goes for
source and object files.)  Also, if you try to import multiple files with
the same basename (e.g. ``bar.o'' and ``bar.so'') the behavior is
currently undefined.





\section{Foreign Sources}

A call to register a foreign source has the same form as for a foreign
function: ``{\tt foreign\_source({\em function-name}, {\em
file-list})}''.  However, in this case the {\em function-name} is the
name of the function being {\em exported}.  The call to {\tt
foreign\_source} will return a stream of incoming values.  It must be
annotated with a type of the form {\tt Stream} $T$, where $T$ is a
type that supports marshaling from C code.

We call the function exported to C an {\em entrypoint}.  When called
from C, it will take a single argument, convert it to the WaveScript
representation, and fire off a tuple as one element of the input
stream.  The return behavior of this entrypoint is determined by the
scheduling policy employed by that particular WaveScope backend.  For
example, it may follow the tuple through a depth-first traversal of
the stream graph, returning only when there is no further processing.
Or the entrypoint may return immediately, merely enqueuing
the tuple for later processing.  The entrypoint returns an integer
error code, which is zero if the WaveScope process is in a
healthy state at the time the call completes.  Note that a zero
return-code does not guarantee that an error will not be encountered
in the time between the call completion and the next invocation of the
entrypoint. 

Currently, using multiple foreign sources is supported (i.e. multiple
entrypoints into WaveScript).  However, if using foreign sources, you
cannot also use built-in WaveScript ``timer'' sources.  When driving
the system from foreign sources, the entire WaveScript system becomes
just a set of functions that can be called from C.  The system is
dormant until one of these entrypoints is called.

Because the main thread of control belongs to the foreign C code,
there is another convention that must be followed.  The user must implement
{\em three} functions that WaveScript uses to initialize, start up the
system, and handle errors respectively.

\begin{verbatim}
  void wsinit(int argc, char** argv)
  void wsmain(int argc, char** argv)
  void wserror(const char*)
\end{verbatim}

{\tt Wsinit} is called at startup, before any WaveScript code runs
(e.g. before {\tt state\{\}} blocks are initialized, and even before
constants are allocated).  {\tt Wsmain} is called when the WaveScript
dataflow graph is finished initialing and is ready to receive data.
{\tt Wsmain} should control all subsequent acquisition of data, and
feed data into WaveScript through the registered {\tt foreign\_source}
functions.  {\tt Wserror} is called when WaveScope reaches an error.
This function may choose to end the process, or may return control to
the WaveScope process.  The WaveScope process is thereafter
``broken''; any pending or future calls to entrypoints will return a
non-zero error code.

\section{Inline C Code}

The function for generating and including C code in the compiler's
output is {\tt inline\_C}.  We want this bso that we can
{\em generate} new/parameterized C code (by pasting strings together) rather than 
including a static {\tt .c} or {\tt .h} file, and instead of using
some other mechanism (such as the C preprocessor) to generate the C code.
The function  is called as ``{\tt inline\_C({\em c-code},
{\em init-function})}''.  Both of its arguments are strings.  The first
string contains raw C-code (top level declarations).  The second
argument is either the null string, or is the name of an
initialization function to add to the list of initializers called
before {\tt wsmain} is called (if present).  This method enables us to
generate, for example, an arbitrary number of C-functions dependent on
an arbitrary number of pieces of global state.  Accordingly we also
generate initializer functions for the global state, and register them
to be called at startup-time.

The return value of the {\tt inline\_C} function is a bit odd.  It
returns an empty stream (a stream that never fires).  This stream may
be of any type; just as the empty list may serve as a list of any
type.  This convention is an artifact of the WaveScript metaprogram
evaluation.  The end result of metaprogram evaluation is a dataflow
graph.  For the inline C code to be included in the final output of
the compiler, it must be included in this dataflow graph.  Thus {\tt inline\_C}
returns a ``stream'', that must in turn be included in the dataflow
graph for the inline C code to be included.
%
You can do this by using the {\tt merge} primitive to combine it with any
other Stream (this will not affect that other stream, as {\tt
inline\_C} never produces any tuples).  Alternatively, you can return the
output of {\tt inline\_C} directly to the ``main'' stream,
as follows:

\begin{wscode}
main = inline\_C(\dots)
\end{wscode}

%% Note that a WaveScript program can contain multiple ``{\tt BASE <-}''
%% statements that are implicitly combined with the {\tt merge}
%% primitive.  Of course, all the streams returned in this way must be of
%% the same type.
%% %
%% \rednote{Actually, this feature is planned but not implemented at the
%%   moment [2007.08.24].}


\section{Foreign code in TinyOS}\label{s:foreigntos}

There are a few differences in how foreign code works in TinyOS. The {\cde
foreign\_source} function is virtually the same.  The {\cde foreign} function is similar,
but it should be noted that one can cheat a little by suppying an
arbitrary string for the function name.  For example, here is a
foreign function that toggles an LED, written using NesC's {\tt call} syntax:

\begin{verbatim}
led0Toggle = 
  (foreign("call Leds.led0Toggle", []) 
   :: () -> ());
\end{verbatim}

\rednote{Currently [2008.02.22] foreign only works for functions
  returning {\cde void}.  Bug me to fix this.}

The major difference lies in {\cde inline\_C}, which is replaced by
{\cde inline\_TOS}.  TinyOS simply has more places that one might want
to stick code, thus more hooks are exposed:

\begin{wscode}
inline\_TOS({\em top}, {\em conf1}, {\em conf2}, {\em mod1}, {\em mod2}, {\em boot})
\end{wscode}

All arguments are strings.  They inject code into different contexts
as follows:
\begin{enumerate}
\item {\bf top}: Inject code at top-level, not iside a {\em configuration}
  or {\em module} block.
\item {\bf conf1}: Inject code into the {\cde configuration} block produced
  by the WaveScript compiler.
\item {\bf conf2}: Inject code into the {\cde implementation} section of
  that {\cde configuration} block.
\item {\bf mod1}: Inject code into the {\cde module} block produced
  by the WaveScript compiler.
\item {\bf mod2}: Inject code into the {\cde implementation} section of
  that {\cde module} block.
\item {\bf boot}: Inject code into the {\cde Boot.booted()} event handler.
\end{enumerate}

This mechanism for inlining NesC code can be used for adding support
for new timers or data sources.  In fact, this is how existing
 functions like {\cde tos\_timer} and {\cde sensor\_uint16}
are implemented.  (See {\cde internal\_wstiny.ws} inside the {\tt
  lib/} directory.)

\section{Other backend's support}

The foreign interface works to varying degrees under each backend.
%the {\bf Scheme}, {\bf MLton}, and {\bf C++/XStream} backends ({\tt
%  ws}, {\tt wsmlton}, and {\tt wsc} respectively).
Below we discuss the current  limitations in each
 backend.  The feature matrix in Figure \ref{f:features} gives an
 overview.

\rednote{This has not been updated to address {\bf wsc2}.}

\begin{figure}
\begin{center}
\begin{tabular}{|r|r|r|l|}
\hline $feature$ & $ws$ & $wsmlton$ & $wsc$ \\
\hline

{\tt foreign}         & yes     & yes & yes \\
{\tt foreign\_source} & never   & yes & not yet \\
{\tt inline\_C}       & not yet & yes & not yet \\

loads {\tt .c}        & yes     & yes     & yes \\
loads {\tt .h}        & yes     & yes     & yes \\
loads {\tt .o}        & yes     & not yet & yes \\
loads {\tt .a}        & yes     & not yet & yes \\
loads {\tt .so}       & yes     & no      & yes \\


marshal scalars    & yes     & yes      & yes \\
marshal arrays     & no      & yes      & not yet \\
{\tt ptrToArray}   & no      & yes      & not yet \\
{\tt exclusivePtr} & yes     & not yet  & yes \\

\hline
\end{tabular}
\end{center}
\caption{Feature matrix for foreign interface in different backends}
\label{f:features}
\end{figure}

Note that even though the Scheme backend is listed as supporting {\tt
.a} and {\tt .o} files, the semantics are slightly different than for
the C and MLton backends.  The Scheme system can only load
shared-object files, thus when passed {\tt .o} or {\tt .a} files, it
simply invokes a shell command to convert them to shared-object files
before loading them.

Including source files also has a slightly different meaning between
the Scheme and the other backends.  Scheme will ignore header files
(it doesn't need them).  Then C source files ({\tt .c} or {\tt .cpp})
are compiled by invoking the system's C compiler.  On the other hand,
in the XStream backend, C source files are simply {\tt \#include}d
into the output C++ query file.  In the former case, the source is
compiled with no special link options or compiler flags, and in the
latter it is compiled under the same environment as the C++ query file
itself.  

Thus the C source code imported in this way must be fairly
robust to the {\tt gcc} configuration that it is called with.
%  If
%greater control is needed it is recommended to precompile the C code
%into a library or object file.
If the imported code requires any customization of
the build environment whatsoever, it is recommended to compile them
yourself and import the object files into WaveScript, rather than
importing the source files.

\rednote{[2007.05.03] Note: Currently the foreign function interface is only
supported on Linux platforms.  It also has very preliminary support
for Mac OS but has a long way to go.}




\section{Converting WaveScript and C types}

An important feature of the foreign interface is that it defines a set
of mappings between WaveScript types and native C types.  The compiler
then automatically converts, where necessary, the representation of arguments to foreign
functions.
This allows many C functions to be used without modification, or ``wrappers''.  Figure
\ref{f:types} shows the mapping between C types and WaveScript types.

\rednote{[2007.08.24] Currently wsmlton does not automatically null
  terminate strings.  This needs to be fixed, but in the meantime the
  user must null terminate them manually.}

\begin{figure}
\begin{center}
\begin{tabular}{|r|r|l|}
\hline
$WaveScript$ & $C$ & $explanation$\\
\hline
{\tt Int}   & {\tt int}   & 
  \parbox[t]{2.2in}{native ints have a system-dependent 
length, note that in the Scheme backend WaveScript {\tt Int}s may 
have less precision than C {\tt int}s} \\

{\tt Float} & {\tt float} & 
\parbox[t]{2.2in}{WaveScript floats are single-precision}\\

{\tt Double} & {\tt double} & \\

{\tt Bool} &   {\tt int} & \\

{\tt String} & {\tt char*} & pointer to null-terminated string \\

%()   & 

{\tt Char} & {\tt char} &  \\

{\tt Array T} & {\tt $T$*} & \parbox[t]{2.2in}{
pointer to C-style array of elements of type {\tt T}, where {\tt T}
must be a scalar type
}\\

{\tt Pointer} & {\tt void*} &  \parbox[t]{2.2in}{
 Type for handling C-pointers.  Only good for
  passing back to C.
}\\

\hline
\end{tabular}
\end{center}
\caption{Mapping between WaveScript and C types.  Conversions
  performed automatically.}
\label{f:types}
\end{figure}

\rednote{[2007.05.03] The system will very soon support conversion of
  Complex and Int16 types.  
%Further, it might provide a
%  C-library for manipulating the representations of other WaveScript
  types.
}


\section{Importing C-allocated Arrays}

A WaveScript array is generally a bit more involved than a C-style
array.  Namely, it includes a length field, and potentially other
metadata.  In some backends ({\tt wsc}, {\tt wsmlton}) it is easy to
pass WaveScript arrays to C without copying them, because the WS array
contains a C-style array within it, and that pointer may be passed
directly.

Going the other way is more difficult.  If an array has been allocated
(via {\tt malloc}) in C, it's not possible to use it directly in
WaveScript.  It lacks the necessary metadata and lives outside the
space managed by the garbage collector.  However, WaveScript does
offer a way to {\em unpack} a pointer to C array into a WaveScript
array.  Simple use the primitive {\tt ``ptrToArray''}. But, as with
foreign functions, be sure to include a type annotation.  (See the
table in Figure \ref{f:features} for a list of backends that currently
support {\tt ptrToArray}.)

\section{``Exclusive'' Pointers}

Unfortunately, {\tt ptrToArray} is not always sufficient for our
purposes.  When wrapping an external library for use in WaveScript, it
is desirable to use memory allocated outside WaveScript, while
maintaining a WaveScript-like API.  For instance, consider a Matrix
library based on the Gnu Scientific Library (GSL), as will described in the
next chapter.  GSL matrices must be allocated outside of WaveScript.
Yet we wish to provide a wrapper to the GSL matrix operations that
feels natural within WaveScript.  In particular, the user should not
need to manually deallocate the storage used for matrices.

For this purpose, WaveScript supports the concept of an {\em
  exclusive} pointer.  ``Exclusive'' means that no code outside of
WaveScript holds onto the pointer.  Thus when WaveScript is done
with the pointer the garbage collector may invoke {\tt free} to
deallocate the referenced memory.  (This is equivalent to calling {\tt
  free} from C, and will not, for example, successfully deallocate a
  pointer to a pointer.)

Using exclusive pointers is easy.  There is one function {\tt
  exclusivePtr} that converts a normal {\tt Pointer} type (machine
  address) into a managed exclusive pointer.  By calling this, the
  user guarantees that that copy of the pointer is the only in
  existence.  Converting to an exclusive pointer should be thought of
  as ``destroying'' the pointer---it cannot be used afterwards.  To
  retrieve a normal pointer from the exclusive pointer, use the {\tt
  getPtr} function.  \rednote{In the future, getting an exclusive
  pointer will ``lock'' it, and you'll have to release it to make it
  viable for garbage collection again.  Currently, this mechanism is
  unimplemented, so you must be careful that you use the {\tt Pointer}
  value that you get back immediately, and that there is a live copy
  of the exclusive pointer in existence to keep it from being free'd
  before you finish with it.}


\chapter{Standard Libraries}

\section{{\tt stdlib.ws}}

Most of the basic stream operators (window, rewindow, sync, etc) are
contained within ``stdlib.ws''.  Thus most programs will start with
{\tt include ``stdlib.ws''}.  The file is contained within the {\tt
  lib/} subdirectory of the WaveScript root.
The top portion of the file contains type signatures for all the
functions exported by the library.  Further down, above each
definition, should be a brief explanation of its function.

Here is a high level overview of the functionality provided by
{\tt stdlib.ws} as of [2007.08.24]:

\begin{itemize}
\item Stream operators: snoop on streams, zip streams together, sync them, window,
  rewindow, and dewindow them.  Interleave and deinterleave the
  elements of streams.
\item Sigseg operators: map functions over them, lift some operations
  like fft to work on sigsegs.
\item Basic math functions that are not included as primitive.
\item Prints, asserts, and constant definitions (e.g. $\pi$).
\item Extended list and array operations not included as primitive (e.g. {\tt List:map2}).
\item Shorthands for common procedures (e.g. {\tt i2f} for {\tt intToFloat}).
\item Curried versions of higher order operators.
\end{itemize}


\section{Fourier Transforms}

WaveScript uses fftw.  See prim\_defs.ss for a list of the different
fft interfaces provided.




\section{Matrix Operations}

WaveScript uses the Gnu Scientific Library to support matrix
operations.  There are three files of interest within the {\tt
wavescript/lib/} directory.  The first, {\tt gsl.ws} declares
prototypes for accessing a subset of the low level GSL functions
directly.  You shouldn't need to use this file directly.  The second,
{\tt matrix\_gsl.ws}, which you should use, provides a wrapper around
GSL's matrix functionality that's more in the spirit of WaveScript.
(Note that both of these files are generated by the C-preprocessor from
the corresponding {\tt .pp} files.)

The third file of interest is {\tt matrix.ws}.  This is a
native-WaveScript library that implements the basic matrix operations
described below using a simple array-of-arrays representation.  Except
for a few operations implemented only in the GSL-based version,
the libraries should be interchangeable.  You should choose which to
use based on the operations you need, availability of GSL on the
target platform, and performance requirements.

\subsection{Other numeric types}

All of the below matrix operations are explained with the example of
32-bit floating point matrices.  These operators are contained in the
namespace {\tt Matrix:Float} and can may be referred to with fully
qualified names ({\tt Matrix:Float:add}) or by first importing the
namespace, ``{\tt using Matrix:Float};'', followed by only the short
name ``{\tt get}''.  Also included in the matrix library are analogous
routines in the {\tt Matrix:Complex} and {\tt Matrix:Double} namespaces.
\footnote{\rednote{(Currently, the {\tt invert} operation is available only for
{\tt Matrix:Double}.)}}

\subsection{Matrix Interface}

Here are the functions contained within the namespace {\tt
  Matrix:Float}.  On the left is a typical function call, with
  meaningful names for the arguments, and on the right is the
  WaveScript type of the function.  For brevity ``{\tt M}'' is used to
  abbreviate the type {\tt Matrix Float}, which in turn is a type
  alias for whatever the internal representation of a matrix is.

\[
\begin{array}{lcl}
\bullet$ $ $create$(dim1,dim2)     & :: &  $(Int, Int)                \ra   M$   \\
\bullet$ $ $eq$(m1, m2)            & :: &  $(M, M)          \ra   Bool$   \\
\bullet$ $ $get$(m, dim1, dim2)    & :: &  $(M, Int, Int)        \ra   Float$   \\
\bullet$ $ $set$(m, i, j, val)     & :: &  $(M, Int, Int, Float) \ra   ()$ \\
\bullet$ $ $dims$(m)               & :: &  $ M                   \ra   (Float * Float)$ \\
\bullet$ $ $row$(m,i)              & :: &  $ (M, Int)            \ra   Array Float$ \\
\bullet$ $ $col$(m,j)              & :: &  $ (M, Int)            \ra   Array Float$ \\
\bullet$ $ $toArray$(m)            & :: &  $ M                   \ra   Array Float $ \\
\bullet$ $ $fromArray$(arr,dim1)   & :: &  $ (Array Float, Int)       \ra   M $ \\
\bullet$ $ $fromArray2d$(arr,dim1) & :: &  $ (Array (Array Float))    \ra   M $ \\
\bullet$ $ $fromList2d$(arr,dim1)  & :: &  $ (List (List Float))    \ra   M $ \\
\end{array}
\]

This basic interface includes functions for creating a (zeroed)
matrix, accessing and mutating it, and converting to and from
one-dimensional (row-major) arrays, or from two-dimensional arrays and
lists.  A very important note about Array extraction operators such as
{\tt toArray}, {\tt row}, and {\tt column} is that they provide {\em
no guarantee} as to whether or not they return an alias to the
existing storage in the matrix, or newly allocated storage.  We need
to leave both these possibilities open because of the diversity of
possible backends, platforms, and matrix implementations.  Thus the
Arrays returned from these operations must be treated as {\em
immutable}.

In addition to the above basic matrix functions, the matrix library
namespace also includes common linear algebra matrix operations.

\[
\begin{array}{lcl}
\bullet$ $ $add$(m1, m2)           & :: &  $(M, M)          \ra   M$ \\
\bullet$ $ $sub$(m1, m2)           & :: &  $(M, M)          \ra   M$ \\
\bullet$ $ $mul\_elements$(m1, m2) & :: &  $(M, M)          \ra   M$ \\
\bullet$ $ $div\_elements$(m1, m2) & :: &  $(M, M)          \ra   M$ \\
\bullet$ $ $scale$(m, coef)        & :: &  $(M, Float)           \ra   M$ \\
\bullet$ $ $add\_constant$(m1, const) & :: &  $(M, Float)        \ra   M$ \\
\bullet$ $ $mul$(m)                & :: &  $ (M, M) \ra M  \ra   M$ \\
\bullet$ $ $invert$(m)             & :: &  $ M                   \ra   M$ \\
\end{array}
\]

The above operations are purely functional.  That is, they do not
destroy their arguments; they allocate new matrices to store the
results of their computations.  Because matrices can be large, this is
not always desirable.  The matrix library includes destructive, {\em
in place} versions of all the above operations (except {\tt mul} and
{\tt invert}): for example, {\tt add\_inplace}.  These mutate their
first argument and return unit, ``{\tt ()}'', rather than returning a
new matrix.
%They have the suffix ``\_inplace'' append to their names.

\subsubsection{Higher order matrix operations}

Below are additional {\em higher order} matrix operations (those that
take functions as arguments).

\[
\begin{array}{lcl}
\bullet$ $ $build$    & :: & $(Int, Int, (Int, Int) \ra Float) \ra M  $ \\
\bullet$ $ $foreach$  & :: & $(Float \ra (), M) \ra ()   $ \\
\bullet$ $ $foreachi$     & :: & $((Int, Int, Float) \ra (), M) \ra ()   $ \\
\bullet$ $ $rowmap$       & :: & $(Array Float \ra b, M) \ra Array b     $ \\
\bullet$ $ $map$          & :: & $(Float \ra Float, M) \ra M  $ \\
\bullet$ $ $map2$         & :: & $((Float,Float) \ra Float, M, M) \ra M  $  \\
\bullet$ $ $map\_inplace$  & :: & $(Float \ra Float, M) \ra ()           $ \\
\bullet$ $ $map2\_inplace$ & :: & $((Float,Float) \ra Float, M, M) \ra ()    $\\
\end{array}
\]

The {\tt build}, {\tt foreach}, {\tt foreachi}, and {\tt map}
operations follow the same conventions as the standard WaveScript
container operators of the same names.  For example, {\tt
build($rows$, $cols$, $f$)} builds a matrix of the specified size by
applying $f$ at every $(i,j)$ index pair.  Additionally, {\tt rowmap}
exposes each individual row as an array, and the ``inplace'' map
variants provide destructive updates.

\subsubsection{A note on polymorphism}

The operations contained in {\tt Matrix:Float} are {\em monomorphic}.
They operate only on matrices of floats.  This is due to the fact that
these are wrappers around non-polymorphic native C routines.  While
the pure WaveScript matrix implementation provides these monomorphic
interfaces for compatibility, it also provides polymorphic operations
directly under the {\tt Matrix} namespace (i.e. {\tt Matrix:add}).
There are definite advantages to the polymorphic interface.  For
example, the {\tt Matrix:map} function can be used to apply a function
$f :: $Float$ \ra $Complex to build a complex matrix from a float
matrix.  The monomorphic versions can only apply map float matrices to
float matrices and complex to complex.

\subsubsection{Going further: other GSL functionality}

The GSL libraries contained a wealth of functionality not currently
exposed in WaveScript.  It is straightforward to extend {\tt gsl.ws}
to export more of this functionality, but there has not been the
opportunity or need to do so at the moment.  You can see the GSL
documentation here:

\vspace{-2mm}
\begin{center}
\url{http://www.gnu.org/software/gsl/manual/html_node/}
\end{center}

If there's something you need, bug Ryan to add
it (\url{newton@mit.edu}), or take a look at {\tt gsl.ws.pp} and {\tt matrix\_gsl.ws.pp} and
try to add it yourself.



%================================================================================



\chapter{WaveScript Evaluation Model}
\label{s:evalmodel}

WaveScript is a {\em metaprogramming} language.  For further
explanation, I'll refer you to a quote from the {\em Matrix Reloaded}
that was brought to my attention by Yannis Smaragdakis.

\begin{center}
NEO: Programs hacking programs. Why?

ORACLE: They have their reasons, ...
\end{center}

The {\em reason} in WaveScript, is that we want to write programs of a
different character than we want to {\em run}.  We call this {\em asymmetric metaprogramming}.
We want to {\em write}
abstract, reusable, polymorphic, higher-order programs, but we want to
{\em run} parallelizable dataflow graphs where individual nodes are
fast, monomorphic, first-order, C-programs that don't require
a heavyweight runtime.
In WaveScript your program essentially {\em
generates} the specialized, high-performance stream-processing
program that is subsequently compiled and run.  


How does this work?  
WaveScript runs your program through an interpreter at compile time.
Your program returns a stream value.  WaveScript must then make sure
that it can convert that stream-value back to code, and reduce that
code to fit the restrictions of the backend.  There are restrictions
on the backend because we omit heavyweight language features to target
tiny embedded devices (e.g. no functions as values / no closures).
Hence {\em asymmetric} metaprogramming.

Your is therefore really two programs: a {\em meta-}program and an {\em
  object}-program.  The meta-program runs at compile-time and
  generates the object-program.  Other metaprogramming systems
  typically have heavyweight quotation mechanisms for explicitly
  constructing object code, so you know exactly what object-code you
  get out.  However, these mechanisms are also very onerous, and place
  a burden on the programmer.  

WaveScript intentionally blurs the distinction between the
stages---uses the same language, the same syntax, semantics, and
libraries for both stages---and most of the time you don't need to
think about it.  But sometimes you do.  In particular, you have to be
careful not to try to store function values in data structures within
object code (inside {\cde iterates}).  Recursive functions are also
currently disallowed in object code.


%% What results is a dataflow graph of {\tt iterate} blocks.
%% Thus, only bounded-recursions are permitted in user defined functions,
%% and excessive code bloat becomes a real possibility.  In the future, we will
%% allow a ``noinline'' or ``library function'' annotation that specifies
%% that a function is to be compiled as a C-function, rather than
%% inlined.  Of course, such a function will not be allowed to take or
%% return {\tt Stream} values.

%% \rednote{\em \bf Keep the metaprogram pure for now! (free of side-effects)}

%\chapter{WaveScript Execution Model}
\section{Execution Model}

\rednote{UNFINISHED}

After the metaprogram evaluates, what is left is a dataflow graph of
stream operators (kernels).  Now you may ask: what are the semantics
of this dataflow graph's execution?  WaveScript implements an {\em
asynchronous dataflow} model.  Streams are sequences discrete events,
with no implied synchronization.  Stream kernels are thus
event-handlers.  Further, the order of evaluation of kernels within a
data flow graph is entirely at the whim of the scheduler, subject to
loose fairness constraints.  Any kernel with input data available is
{\em ready} and may execute at any time; yet no ready kernel may
execute an infinite number of times without all other ready kernels
executing.

%\rednote{Other guarantees?  There was supposed to be one more...}

%% Of course, in practice we have slightly stronger expectations of the
%% scheduler.  We expect that it is reasonably well engineered, and does not
%% follow perverse practices regarding fairness.  For example,
%% depth-first traversals

Therefore computations must be robust to the timing of the
communication channels between kernels.  After all, the program may be
distributed, with some channels going over slow wireless links, and
other channels intra-node.  How do we accomplish this?  Well, we rely
on well-engineered libraries to insulate programmers from as much of
the pain of asynchronicity as possible.  In particular, be careful
whenever multiple streams are being merged.  If there is a one-to-one
correspondence between stream elements, use {\cde zip} library
functions if possible.

%% \rednote{TODO: go over and expand this}

%% \section{Debugging}

%% \rednote{TODO}


\appendix
\chapter{Primitive table}\label{s:primtable}

This is an automatically generated table of primitives and their type
signatures.  Also see the libraries in {\cde lib/} for
additional standard functions.

{
%\footnotesize
\scriptsize
\input{primtable.tex}
}


%% pos#0: 
%% seg-get: index 474 is out of bounds for sigseg:
%% #<struct:sigseg>

%%  === context ===
%% /Users/newton/wavescript/src/generic/sim/wavescript_sim_library_push.ss:1314:5: seg-get
%% /Users/newton/temp/__lang_running.tmp.ss:359:30
%% /Users/newton/wavescript/src/generic/sim/wavescript_sim_library_push.ss:357:6: wsbox
%% /Users/newton/wavescript/src/generic/sim/wavescript_sim_library_push.ss:282:13: global-loop



\end{document}
